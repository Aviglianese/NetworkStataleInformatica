%%%%%%%%%%%%%%%%%%%%%%%%%%%%%%%%%%%%%%%%%
% Formal Book Title Page
% LaTeX Template
% Version 2.0 (23/7/17)
%
% This template was downloaded from:
% http://www.LaTeXTemplates.com
%
% Original author:
% Peter Wilson (herries.press@earthlink.net) with modifications by:
% Vel (vel@latextemplates.com)
%
% License:
% CC BY-NC-SA 3.0 (http://creativecommons.org/licenses/by-nc-sa/3.0/)
%%%%%%%%%%%%%%%%%%%%%%%%%%%%%%%%%%%%%%%%%

%----------------------------------------------------------------------------------------
%	PACKAGES AND OTHER DOCUMENT CONFIGURATIONS
%----------------------------------------------------------------------------------------

\documentclass[a4paper, 11pt, oneside, article]{book} % A4 paper size, default 11pt font size and oneside for equal margins

\newcommand{\plogo}{\fbox{$\mathcal{PL}$}} % Generic dummy publisher logo

\usepackage[utf8]{inputenc} % Required for inputting international characters
\usepackage[T1]{fontenc} % Output font encoding for international characters
\usepackage{hyperref}
\usepackage[dvipsnames]{color}
%\usepackage{fourier} % Use the New Century Schoolbook font

%----------------------------------------------------------------------------------------
%	TITLE PAGE
%----------------------------------------------------------------------------------------

\begin{document} 

\begin{titlepage} % Suppresses headers and footers on the title page

	\centering % Centre everything on the title page
	
	\scshape % Use small caps for all text on the title page
	
	\vspace*{\baselineskip} % White space at the top of the page
	
	%------------------------------------------------
	%	Title
	%------------------------------------------------
	
	\rule{\textwidth}{1.6pt}\vspace*{-\baselineskip}\vspace*{2pt} % Thick horizontal rule
	\rule{\textwidth}{0.4pt} % Thin horizontal rule
	
	\vspace{0.75\baselineskip} % Whitespace above the title
	
	{\LARGE REGOLAMENTO\\ NETWORK\\ STATALEINFORMATICA\\} % Title
	
	\vspace{0.75\baselineskip} % Whitespace below the title
	
	\rule{\textwidth}{0.4pt}\vspace*{-\baselineskip}\vspace{3.2pt} % Thin horizontal rule
	\rule{\textwidth}{1.6pt} % Thick horizontal rule
	
	\vspace{2\baselineskip} % Whitespace after the title block
	
	%------------------------------------------------
	%	Subtitle
	%------------------------------------------------
	
	Si consiglia di leggere le regole indicate prima di utilizzare qualsiasi gruppo telegram del network
	\url{https://t.me/stataleinformatica}
	
	\vspace*{3\baselineskip} % Whitespace under the subtitle
	
	%------------------------------------------------
	%	Editor(s)
	%------------------------------------------------
	
	%Edited By
	
	\vspace{0.5\baselineskip} % Whitespace before the editors
	
	{  \scshape 
		Per qualsiasi dubbio contattare\\ 
		gli amministratori del network \break \\ 
		@aconithorn - 3\textordmasculine \hspace{0.01cm} anno\\
		@davidebusolin - 3\textordmasculine \hspace{0.01cm} anno\\
		@giuseppetm - 3\textordmasculine \hspace{0.01cm} anno\\ 
		@georgianiandrei - 2\textordmasculine \hspace{0.01cm} anno\\
		@gbitgbit - 2\textordmasculine \hspace{0.01cm} anno\\
		@mrbrionix - 2\textordmasculine \hspace{0.01cm} anno\\
		@iriscanole - 1\textordmasculine \hspace{0.01cm} anno\\
		@mantotheale - 1\textordmasculine \hspace{0.01cm} anno\\
		@mattia\_oldani - 1\textordmasculine \hspace{0.01cm} anno\\
		@weinsz - 1\textordmasculine \hspace{0.01cm} anno\\
		} % Admin list
	
	\vspace{3\baselineskip} % Whitespace below the editor list
	
	È possibile accedere al drive \\del network al seguente link\\
	\textbf{\href{https://drive.google.com/drive/folders/0BwzuyD3iLGcbcUNxTVNOVE9FR1E}{drive}}
	
	%\textit{The University of California \\ Berkeley} % Editor affiliation
	
	\vfill % Whitespace between editor names and publisher logo
	
	%------------------------------------------------
	%	Publisher
	%------------------------------------------------
	
	{\large Ultima modifica in data 23/09/2020} 

\end{titlepage}

%----------------------------------------------------------------------------------------

%----------------------------------------------------------------------------------------
%	REGOLAMENTO
%----------------------------------------------------------------------------------------

\textcolor{red}{\ttfamily{\textsl{{\Large Perchè abbiamo introdotto un regolamento?}}}}\\

Vogliamo rendere chiari i motivi per cui abbiamo deciso di regolamentare i gruppi del nostro network.
Abbiamo notato che la maggior parte di essi erano tempestati di domande banali, fatte più volte al giorno, la cui risposta era facilmente trovabile. 
Questo riduce la qualità della chat e scoraggia la partecipazione di studenti più attenti. Per questo motivo abbiamo deciso di provare a limitare il fenomeno, 
da una parte ammonendo chi continua a fare interventi non produttivi, e dall'altra fornendo un modo facile e veloce per trovare le informazioni più importanti, 
tramite le pagine FAQ di ogni insegnamento.\\

\textbf{Dove trovo le pagine FAQ dei corsi?}\\

Nei prossimi mesi adotteremo una strategia per far sì che ogni corso abbia una pagina relativa alle domande fatte più frequentemente.
Per il momento sarebbe l'ideale limitare comunque il fenomeno delle domande ripetute più volte semplicemente chiedendovi di cercare nelle chat precedenti.
Telegram permette di farlo in modo efficiente e veloce, quindi siete pregati di provare a farlo prima di porre domande; ovviamente nel caso in cui non ci sia la risposta che cercate siete liberi di chiedere in chat.\\\\
Nella prossima pagina è possibile trovare il regolamento.

\newpage

\begin{enumerate}
\rmfamily
\item {\ttfamily{\textsl{{\Large Regole riguardo domande e faq}}}}\\
	
	\textbf{Domanda con risposta indicata sul sito del docente}\\
	L'utente riceve un warn con successiva indicazione su dove trovare l'informazione richiesta.
	Dopo 3 warn l'utente viene mutato per un giorno, a discrezione degli amministratori.\\
	
	\textbf{Domanda riguardo informazioni banali e già chieste\\ precedentemente nella chat}\\
	Nessuna penalità se non sono presenti le FAQ del gruppo del relativo corso, altrimenti vale la regola precedente, ovvero: l'utente riceve un warn con successiva 					indicazione della presenza delle FAQ. Dopo 3 warn viene mutato per un giorno, a discrezione degli amministratori.\\
	È ovviamente consigliato (nel caso non ci fossero ancora le FAQ di quel corso) di controllare i messaggi precedenti per scoprire se è già stata data risposta alla vostra domanda.
	
\rule{\linewidth}{0.1mm}

\item  {\ttfamily{\textsl{{\Large Regole comportamentali}}}}\\

	\textbf{Bestemmie e linguaggio vivace}\\
	Entrambi permessi finchè non diventa spam: in quel caso l'utente verrà mutato per un certo periodo di tempo a discrezione degli amministratori.\\
	
	\textbf{Insulti e offese}\\
	Sono entrambi vietati, pena warn per gli insulti leggeri, mute o ban per quelli più pesanti, a discrezione degli amministratori.\\
	Una nota riguardo i docenti: si può benissimo criticare (nei limiti del normale) ma non si accettano assolutamente insulti verso di essi.\\
	
	\textbf{Spam e offtopic}\\
	Lo spam è vietato.\\
	Le conversazioni offtopic sono vietate nei gruppi specifici dei corsi, mentre sono accettate nei gruppi generali degli anni accademici (a patto che non diventino eccessive).

\rule{\linewidth}{0.1mm}
	
\end{enumerate}




\end{document}
