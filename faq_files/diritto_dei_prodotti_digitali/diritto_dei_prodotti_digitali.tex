\documentclass{article}

\usepackage[utf8]{inputenc}
\usepackage[T1]{fontenc} 
\usepackage{hyperref}
\usepackage[dvipsnames]{color}
\usepackage[nodayofweek,ddmmyyyy]{datetime}


% Defines a new environment, which for now it's just a simple list. The first arguments are the actions to 
% take at \begin{QuestionList}, the other one the ones to take at the \end
% 
% For enumerate customization, please refer to http://www.texnia.com/archive/enumitem.pdf

\newenvironment{QuestionList}[1][]
  {\begin{enumerate}[itemsep=10pt,parsep=10pt]}
  {\end{enumerate}}
    



% Defines a new command, which represents a question/answer pair. The first argument is the questions, while the second one is the answer to that question. Please note that this command should be used inside a "QuestionList", so its rules (e.g. paragraph spacing) apply.

\newcommand{\question}[2] {
		\item{\ttfamily{\textsl{\large{#1}}}}
		
    	{#2}
}




\title{FAQ - \textbf{Diritto dei prodotti digitali}}
\author{
	Corso complementare (scelta libera)\\6 CFU, Secondo semestre\\
	Docente: Giorgio Pedrazzi\\ 
	\href{https://gpedrazzidpd.ariel.ctu.unimi.it/v5/home/default.aspx/}{Sito web}
	\date{}
}

\begin{document} 

\maketitle
	
\begin{QuestionList}
		
    \question{Dove si trova il sito web del corso?} {
        Si trova qui sopra cliccando "sito web".
	}
		
	\question{Non riesco ad accedere al sito Ariel, cosa devo fare?} {
		Si presume che tu sia di un altro corso di laurea diverso da comunicazione digitale e voglia inserire
		questo corso come complementare a scelta libera. In tal caso, basta scrivere una mail al docente chiedendo di poter accedere al sito Ariel del corso.
	}
		
		\question{Com'è strutturato l'esame?} {
		    L'esame consiste in uno scritto di 20 domande a risposte chiuse, sia in presenza che da remoto.
		}
		
		\question{Qual è il materiale a disposizione per studiare?} {
		    Sono disponibili le \href{https://www.youtube.com/channel/UCJRWh9Jp5G-_f8jvNtvtnwQ}{videolezioni del corso} e le slide che potete trovare sul sito Ariel, insieme ai libri suggeriti dal docente.
		}
		
	\end{QuestionList}
	
\end{document}
