\documentclass{article}

\usepackage[utf8]{inputenc}
\usepackage[T1]{fontenc} 
\usepackage{hyperref}
\usepackage[dvipsnames]{color}
\usepackage[nodayofweek,ddmmyyyy]{datetime}


% Defines a new environment, which for now it's just a simple list. The first arguments are the actions to 
% take at \begin{QuestionList}, the other one the ones to take at the \end
% 
% For enumerate customization, please refer to http://www.texnia.com/archive/enumitem.pdf

\newenvironment{QuestionList}[1][]
  {\begin{enumerate}[itemsep=10pt,parsep=10pt]}
  {\end{enumerate}}
    



% Defines a new command, which represents a question/answer pair. The first argument is the questions, while the second one is the answer to that question. Please note that this command should be used inside a "QuestionList", so its rules (e.g. paragraph spacing) apply.

\newcommand{\question}[2] {
		\item{\ttfamily{\textsl{\large{#1}}}}
		
    	{#2}
}




\title{FAQ - \textbf{Programmazione II}}
\date{}
\author{
	2$^{\circ}$ anno\\6 CFU, Primo semestre\\
	Docente: Massimo Santini\\ 
	\href{https://prog2.di.unimi.it/}{Sito web}
}

\begin{document} 
	\maketitle
	
	\begin{enumerate}
		
		\rmfamily
		\domanda{Dove si trova il sito web del corso?} 
		Lo trovi qui sopra se premi "sito web".\\
		
		\domanda{Com'è strutturato l'esame?} 
		Al momento l'esame a distanza è strutturato nel seguente modo: c'è una prova pratica in cui si accede
		in un ambiente con Visual Studio Code per svolgere il tema d'esame proposto. Se viene superata la prova
		con almeno la sufficienza si accede all'orale, che consiste in una breve correzione della prova svolta
		e alcune domande di teoria che si concentrano sugli argomenti dei libri proposti (Liskov e Bloch) e delle
		spiegazioni fatte a lezione.\\
		
		
		\domanda{Qual è il materiale a disposizione per studiare?} 
		Avete i due libri citati prima su cui basarvi: quello della Liskov è fondamentale, e si consiglia
		di leggere anche il Bloch.\\
		
	\end{enumerate}
	
\end{document}
