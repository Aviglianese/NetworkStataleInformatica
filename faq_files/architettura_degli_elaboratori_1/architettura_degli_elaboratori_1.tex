\documentclass{article}

\usepackage[utf8]{inputenc}
\usepackage[T1]{fontenc} 
\usepackage{hyperref}
\usepackage[dvipsnames]{color}
\usepackage[nodayofweek,ddmmyyyy]{datetime}


% Defines a new environment, which for now it's just a simple list. The first arguments are the actions to 
% take at \begin{QuestionList}, the other one the ones to take at the \end
% 
% For enumerate customization, please refer to http://www.texnia.com/archive/enumitem.pdf

\newenvironment{QuestionList}[1][]
  {\begin{enumerate}[itemsep=10pt,parsep=10pt]}
  {\end{enumerate}}
    



% Defines a new command, which represents a question/answer pair. The first argument is the questions, while the second one is the answer to that question. Please note that this command should be used inside a "QuestionList", so its rules (e.g. paragraph spacing) apply.

\newcommand{\question}[2] {
		\item{\ttfamily{\textsl{\large{#1}}}}
		
    	{#2}
}




\title{FAQ - \textbf{Architettura degli elaboratori I}}
\author{
	1$^{\circ}$ anno\\6 CFU, Primo semestre\\
	Docenti: Alberto Borghese, Nicola Basilico\\ 
	\date{}
}

\begin{document} 

\maketitle
	
\begin{QuestionList}
		
    \question{Dove si trova il sito web del corso?} {
        I siti web del corso sono due per la parte di teoria, dato che è divisa su due turni in base all’iniziale del cognome. 
        Il primo (del prof. Borghese) si trova al seguente \href{https://aborgheseae1.ariel.ctu.unimi.it/v5/home/Default.aspx}{link}, mentre il secondo (del prof. Basilico) al seguente \href{https://nbasilicoae1.ariel.ctu.unimi.it/v5/home/Default.aspx}{link}.
        È sempre utile il sito personale del Prof. Borghese dove è possibile trovare sia il materiale fornito dal docente sia comunicazioni varie, disponibile al seguente \href{http://ais-lab.di.unimi.it/Teaching/Architettura_I/_Arch_I.html}{link}.
	}

		
	\question{Com'è strutturato l'esame?} {
		L’esame di teoria consiste in una parte scritta e orale. Quella scritta si svolge su piattaforma Exam.net e SEB (si presume a crocette), quella orale su zoom e ci si accede se si è passati la parte scritta. Al momento è molto vaga la situazione quindi fate affidamento al sito del docente per sicurezza, ma dovrebbe essere svolta in questo modo.
		Per la parte di laboratorio invece è richiesto lo svolgimento di un progetto in Logisim seguito poi da un colloquio orale. Il voto di teoria vale 2/3 del voto finale, mentre quello di laboratorio 1/3.
	}
	
	\question{Qual è il materiale a disposizione per studiare?}{
		È possibile trovare la lista dei libri consigliati al seguente \href{https://homes.di.unimi.it/borghese/Teaching/Architettura_I/References_I.rtf}{link}.
		Sono inoltre disponibili le videolezioni sul sito ariel del docente, con i relativi pdf.
	}
		
		
	\end{QuestionList}
	
\end{document}
