\documentclass{article}

\usepackage[utf8]{inputenc}
\usepackage[T1]{fontenc} 
\usepackage{hyperref}
\usepackage[dvipsnames]{color}
\usepackage[nodayofweek,ddmmyyyy]{datetime}


% Defines a new environment, which for now it's just a simple list. The first arguments are the actions to 
% take at \begin{QuestionList}, the other one the ones to take at the \end
% 
% For enumerate customization, please refer to http://www.texnia.com/archive/enumitem.pdf

\newenvironment{QuestionList}[1][]
  {\begin{enumerate}[itemsep=10pt,parsep=10pt]}
  {\end{enumerate}}
    



% Defines a new command, which represents a question/answer pair. The first argument is the questions, while the second one is the answer to that question. Please note that this command should be used inside a "QuestionList", so its rules (e.g. paragraph spacing) apply.

\newcommand{\question}[2] {
		\item{\ttfamily{\textsl{\large{#1}}}}
		
    	{#2}
}




\title{FAQ - \textbf{Tecnologie e linguaggi web}}
\date{Ultima modifica:  \today}
\author{Corso complementare\\6 CFU, Secondo semestre\\
Docente: Paolo Ceravolo\\ 
\href{https://pceravolopwm.ariel.ctu.unimi.it/v5/home/Default.aspx}{Sito web}}

\begin{document} 
\maketitle

\begin{enumerate}

\rmfamily
\domanda{Dove si trova il sito web del corso?} 
Si trova qui sopra se clicchi "sito web".\\
	
\domanda{Com'è strutturato l'esame?} 
C'è un orale diviso in due parti: la prima parte riguarda un paio di domande sulla teoria + svolgimento di esercizi, mentre la seconda parte
è la presentazione del progetto sviluppato e discussione su di esso.\\

\domanda{Qual è il materiale a disposizione per studiare?} 
Ci sono le videolezioni registrate, e pdf per integrare. \\

\domanda{La specifica del progetto va approvata dal docente?} 
Sì, bisogna inviare una bozza al docente per vedere se l'idea è consona ai fini del corso.\\

\domanda{Il progetto è difficile? Ha dei requisiti da rispettare?} 
Il progetto deve rispettare alcuni requisiti specificati sul sito del corso, niente di troppo stringente, è perfettamente in linea con gli obiettivi del corso. Non bisogna sviluppare niente di troppo complicato, basta che rispetti i requisiti.
 
\end{enumerate}




\end{document}
