% Template FAQ del network stataleinformatica

\documentclass{article}
\usepackage[utf8]{inputenc}
\usepackage[T1]{fontenc} 
\usepackage{hyperref}
\usepackage[dvipsnames]{color}
\newcommand{\domanda}[1]{\item{\ttfamily{\textsl{{\large{#1}}}}\\\\}}


\title{FAQ - \textbf{Tecnologie e linguaggi web}}
\date{Ultima modifica:  02/10/2020}
\author{Corso complementare\\6 CFU, Secondo semestre\\
Docente: Paolo Ceravolo\\ 
\href{https://pceravolopwm.ariel.ctu.unimi.it/v5/home/Default.aspx}{Sito web}}

\begin{document} 
\maketitle

\begin{enumerate}

\rmfamily
\domanda{Dove si trova il sito web del corso?} 
Si trova qui sopra se clicchi "sito web".\\
	
\domanda{Com'è strutturato l'esame?} 
C'è un orale diviso in due parti: la prima parte riguarda un paio di domande sulla teoria + svolgimento di esercizi, mentre la seconda parte
è la presentazione del progetto sviluppato e discussione su di esso.\\

\domanda{Qual è il materiale a disposizione per studiare?} 
Ci sono le videolezioni registrate, e pdf per integrare. \\

\domanda{La specifica del progetto va approvata dal docente?} 
Sì, bisogna inviare una bozza al docente per vedere se l'idea è consona ai fini del corso.\\

\domanda{Il progetto è difficile? Ha dei requisiti da rispettare?} 
Il progetto deve rispettare alcuni requisiti specificati sul sito del corso, niente di troppo stringente, è perfettamente in linea con gli obiettivi del corso. Non bisogna sviluppare niente di troppo complicato, basta che rispetti i requisiti.
 
\end{enumerate}




\end{document}
