\documentclass{article}

\usepackage[utf8]{inputenc}
\usepackage[T1]{fontenc} 
\usepackage{hyperref}
\usepackage[dvipsnames]{color}
\usepackage[nodayofweek,ddmmyyyy]{datetime}


% Defines a new environment, which for now it's just a simple list. The first arguments are the actions to 
% take at \begin{QuestionList}, the other one the ones to take at the \end
% 
% For enumerate customization, please refer to http://www.texnia.com/archive/enumitem.pdf

\newenvironment{QuestionList}[1][]
  {\begin{enumerate}[itemsep=10pt,parsep=10pt]}
  {\end{enumerate}}
    



% Defines a new command, which represents a question/answer pair. The first argument is the questions, while the second one is the answer to that question. Please note that this command should be used inside a "QuestionList", so its rules (e.g. paragraph spacing) apply.

\newcommand{\question}[2] {
		\item{\ttfamily{\textsl{\large{#1}}}}
		
    	{#2}
}




\title{FAQ - \textbf{Sistemi operativi}}
\date{Ultima modifica: \today}
\author{
	2$^{\circ}$ anno\\12 CFU, Secondo semestre\\
	Docente: Vincenzo Piuri\\ 
	\href{https://vpiuriso.ariel.ctu.unimi.it/v5/home/Default.aspx}{Sito web}
}

\begin{document} 
	\maketitle
	
	\begin{enumerate}
		
		\rmfamily
		\domanda{Dove si trova il sito web del corso?} 
		Lo trovi qui sopra se premi "sito web".\\
		
		\domanda{Com'è strutturato l'esame?} 
		L'esame consiste in due prove scritte che solitamente vengono svolte nello stesso giorno, ovvero parte A e parte B.
		La parte A tratta gli argomenti del corso fino alla gestione del processore inclusa, mentre la parte B il resto
		del programma. Sono delle domande molto generali sui capitoli del libro.\\
		
		
		\domanda{Qual è il materiale a disposizione per studiare?} 
		Sono disponibili delle videolezioni, altrimenti è consigliato usare il libro.
		Qui invece trovate degli appunti di uno studente che possono essere molto utili ai fini dell'esame: \href{http://www.swappa.it/wiki/Uni/SistemiOperativi}{link}\\
		
	\end{enumerate}
	
\end{document}
