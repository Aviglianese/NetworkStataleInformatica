\documentclass{article}

\usepackage[utf8]{inputenc}
\usepackage[T1]{fontenc} 
\usepackage{hyperref}
\usepackage[dvipsnames]{color}
\usepackage[nodayofweek,ddmmyyyy]{datetime}


% Defines a new environment, which for now it's just a simple list. The first arguments are the actions to 
% take at \begin{QuestionList}, the other one the ones to take at the \end
% 
% For enumerate customization, please refer to http://www.texnia.com/archive/enumitem.pdf

\newenvironment{QuestionList}[1][]
  {\begin{enumerate}[itemsep=10pt,parsep=10pt]}
  {\end{enumerate}}
    



% Defines a new command, which represents a question/answer pair. The first argument is the questions, while the second one is the answer to that question. Please note that this command should be used inside a "QuestionList", so its rules (e.g. paragraph spacing) apply.

\newcommand{\question}[2] {
		\item{\ttfamily{\textsl{\large{#1}}}}
		
    	{#2}
}




\title{FAQ - \textbf{Matematica del continuo}}
\author{
	1$^{\circ}$ anno\\12 CFU, Primo semestre\\
	Docenti: Cecilia Cavaterra, Anna Gori\\ 
	\href{https://ccavaterramc.ariel.ctu.unimi.it/v5/home/Default.aspx}{Sito web}
	\date{}
}

\begin{document} 
	\maketitle
	
	\begin{QuestionList}
		
		\question{Dove si trova il sito web del corso?} {
		     Si trova qui sopra cliccando "sito web".
		 }
		
		\question{Come è strutturato l'esame?} {
		    L'esame consiste in una prova scritta e in una prova orale da svolgersi nello stesso appello. 
		    La prova scritta dura 2 ore ed è formata da due parti distinte (piu' la parte zero sui prerequisiti, che non tutti gli studenti devono sostenere).
		    
		    Piu informazioni su come sono composte le due parti sul \href{https://ccavaterramc.ariel.ctu.unimi.it/v5/home/Default.aspx}{sito del corso}.
		}
		
		\question{Qual è il materiale a disposizione per studiare?} {
		    Sono disponibili le videolezioni con le relative slide, nonchè le esercitazioni.
		    
		    Il libro consigliato dalla docente è: "P. Marcellini e C. Sbordone, Calcolo, Liguori".
		}
		
		\question{Quali sono le risorse consigliate?} {
		    \begin{itemize}
		        \item  \href{https://www.wolframalpha.com/}{Wolfram Alpha}, per controllare le soluzioni agli esercizi
		
		        \item \href{https://www.geogebra.org/graphing}{Geogebra}, per vedere nel dettaglio i grafici delle funzioni
		
		        \item \href{http://vc.di.unimi.it/}{Videolezioni} dei Prof. Tarallo e Cigoli (che seguivano grossomodo lo stesso programma), utili specialmente per le esercitazioni
		
		        \item \href{https://goo.gl/9ADmUc}{Stampati lezioni del Prof. Gobbino} 
		        
		    \end{itemize}
		}
		
	\end{QuestionList}
	
\end{document}
