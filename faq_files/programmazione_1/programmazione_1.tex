\documentclass{article}

\usepackage[utf8]{inputenc}
\usepackage[T1]{fontenc} 
\usepackage{hyperref}
\usepackage[dvipsnames]{color}
\usepackage[nodayofweek,ddmmyyyy]{datetime}


% Defines a new environment, which for now it's just a simple list. The first arguments are the actions to 
% take at \begin{QuestionList}, the other one the ones to take at the \end
% 
% For enumerate customization, please refer to http://www.texnia.com/archive/enumitem.pdf

\newenvironment{QuestionList}[1][]
  {\begin{enumerate}[itemsep=10pt,parsep=10pt]}
  {\end{enumerate}}
    



% Defines a new command, which represents a question/answer pair. The first argument is the questions, while the second one is the answer to that question. Please note that this command should be used inside a "QuestionList", so its rules (e.g. paragraph spacing) apply.

\newcommand{\question}[2] {
		\item{\ttfamily{\textsl{\large{#1}}}}
		
    	{#2}
}




\title{FAQ - \textbf{Programmazione I}}
\author{
	1$^{\circ}$ anno\\12 CFU, Primo semestre\\
	\date{}
}

\begin{document} 
	\maketitle
	
	\begin{enumerate}
		
		\rmfamily
		\domanda{Dove si trovano i siti web del corso?} 
		È possibile trovare gli avvisi, il programma e altre informazioni riguardo il corso al seguente \href{http://boldi.di.unimi.it/Corsi/Inf2020/}{link}.\\
		
		Sito per i cognomi da A a L:
		\href{https://atrentinip.ariel.ctu.unimi.it/}{link}\\\\
		Sito per i cognomi da M a Z:
		\href{https://mcasazzapud.ariel.ctu.unimi.it/}{link}\\
		
		\domanda{Come è strutturato l'esame?}
		L’esame è composto da una prova di programmazione individuale in laboratorio (la prova contiene un esercizio di filtro: gli studenti che non superino il filtro non saranno poi valutati) e una prova scritta. 
		A chi supera entrambe le prove viene proposto un voto (ottenuto come media dei voti delle due prove). 
		Gli studenti con voto proposto compreso fra 21 e 27, possono decidere se verbalizzare il voto proposto o rifiutarlo, sostenendo di nuovo l'esame in uno degli appelli successivi. Gli studenti con voto proposto inferiore a 21 devono sostenere un esame orale per verificare le loro competenze. Gli studenti con voto superiore a 27 possono verbalizzare il voto proposto oppure sostenere un esame orale (finalizzato a migliorare il voto). Le varie parti da cui l'esame è composto vanno necessariamente sostenute nello stesso appello, e in particolare chi pur avendone la possibilità decide di non presentarsi alle verbalizzazioni dovrà sostenere nuovamente l'esame.\\
		
		\domanda{Qual è il materiale a disposizione per studiare?}
		I docenti consigliano i seguenti libri:\\
		- Ivo Balbaert: Programmare in go. Pearson, ISBN 8891909661.\\
		- Alan A. Donovan, Brian W. Kernighan: The Go Programming Language, Addison-Wesley.\\
		
	\end{enumerate}
	
\end{document}
